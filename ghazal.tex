\documentclass{article}

\usepackage{setspace,fontspec,xunicode,bidi,bidipoem}
\newfontfamily\urdufont[Script=Arabic,Scale=1.4]{Jameel Noori Nastaleeq}
\renewcommand\poemcolsepskip{1cm}

\begin{document}

\setRTL
\urdufont
\linespread{1.9}
\thispagestyle{empty}

\begin{center}
{\huge غزل} \\

\vspace{10pt}

{\large (سعید احمد ناطؔق لکھنوی)}
\end{center}

\normalsize

\begin{traditionalpoem}

بجھی ہوئی شمع کا دھواں ہوں اور اپنے مرکز کو جا رہا ہوں & کہ دل کی حسرت تو مٹ چکی ہے اب اپنی حستی مٹا رہا ہوں \\ \\ 
 تیری ہی صورت کے دیکھنے کو بتوں کی تصویریں لا رہا ہوں & کہ خوبیاں سب کی جمع کر کے تیرا تصور بنا رہا ہوں \\ \\ 
 کفن میں خود کو چھپا دیا ہے کہ تجھ کو پردے کی ہو نہ ذہمت & نقاب اپنے لئے بنا کر حجاب تیرا اٹھا رہا ہوں \\ \\ 
 محبت انسان کی ہے فطرت کہاں ہے امکانِ ترکِ الفت & وہ اور بھی یاد آ رہے ہیں میں ان کو جتنا بھلا رہا ہوں \\ \\ 
 زباں پہ لبیک ہر نفس میں جبیں پہ سجدہ ہے ہر قدم پہ & یوں جا رہا بتکدے کو ناطق کہ جیسے کعبے کو جا رہا ہوں \\ \\ 
 
\end{traditionalpoem}
\end{document}